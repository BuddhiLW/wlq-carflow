% Created 2021-08-17 Tue 12:33
% Intended LaTeX compiler: pdflatex
\documentclass[11pt]{article}
\usepackage[utf8]{inputenc}
\usepackage[T1]{fontenc}
\usepackage{graphicx}
\usepackage{grffile}
\usepackage{longtable}
\usepackage{wrapfig}
\usepackage{rotating}
\usepackage[normalem]{ulem}
\usepackage{amsmath}
\usepackage{textcomp}
\usepackage{amssymb}
\usepackage{capt-of}
\usepackage{hyperref}
\date{\today}
\title{}
\hypersetup{
 pdfauthor={},
 pdftitle={},
 pdfkeywords={},
 pdfsubject={},
 pdfcreator={Emacs 27.2 (Org mode 9.4.6)}, 
 pdflang={English}}
\begin{document}

\tableofcontents

\bibliography{article-notes}

\section{Equation of continuity}
\label{sec:orga4c8059}
\begin{equation}
  \begin{aligned}
    \dfrac{\partial{\rho}}{\partial{t}} + \dfrac{\partial{\left( \rho{}v \right)}}{\partial{x}}=0
  \end{aligned}
\end{equation}

\section{Navier-Stokes one-dimensional}
\label{sec:orgcda6a20}
Cited \cite{schlichting2016boundary} abud \cite{Kerner_1993} 

\begin{equation}
\begin{aligned}
\label{eq:NS-n1}
\rho{}\left[\frac{\partial{v}}{\partial{t}} + v\frac{\partial{v}}{\partial{x}} \right] = \dfrac{\partial \left(\mu \frac{\partial{v}}{x} \right)}{\partial{x}} - \dfrac{\partial{p}}{\partial{x}} + X
\end{aligned}
\end{equation}

\begin{equation}
\begin{aligned}
\begin{cases}
\rho : \textrm{Car density}\\
p: \textrm{Local car pressure}
v{}: \textrm{Car instant velocity}\\
\mu{}: \textrm{Viscosity}\\
X: \textrm{Sum of all inner particle interaction forces}
\end{cases}
\end{aligned}
\end{equation}

\section{Mathematical meanings of \(X\) and \(p\)}
\label{sec:org87656dc}
\subsection{Relaxation process meaning}
\label{sec:org5c0294d}
\begin{itemize}
\item If the perception is that current velocity \(v\) is too slow compared
to what can be safely achieved, then X is positive.
\begin{itemize}
\item If the perception is that current velocity \(v\) is too fast and
\end{itemize}
dangerous compared to the traffic condition, then X is negative.
\item Oscillatory behavior can appear.
\end{itemize}

\subsection{Consider the time independent homogeneous condition}
\label{sec:org7448782}
\begin{equation}
\begin{aligned}
\begin{cases}
\label{eq:considerations}
&\langle\frac{\partial{v}}{\partial{x}}\rangle=0 \quad \textrm{(Time independent)} \\
&\langle\dfrac{\partial \left(\mu \frac{\partial{v}}{\partial{x}} \right)}{\partial{x}}\rangle=0 \quad \textrm{(Time independent and Homogeneous)} \\
\\
\quad \because \langle\dfrac{\partial \left(\mu \frac{\partial{v}}{\partial{x}} \right)}{\partial{x}}\rangle &=
\langle\dfrac{\partial \mu}{\partial{x}} \left( \frac{\partial{v}}{\partial{x}}\right)}\rangle +
\langle \mu \left( \dfrac{\partial^2{v}}{\partial^2{x}}\right)}\rangle \\

&\left((\langle\dfrac{\partial \mu}{\partial{x}}\rangle = 0  \quad \textrm{Homogeneous}) \,\land\, (\langle\dfrac{\partial^2{v}}{\partial^2{x}} \rangle = 0 \quad \textrm{Time independent and Homogeneous})\right)\\
&=0\\
\\
&\langle\frac{\partial{p}}{\partial{x}}\rangle=0 \quad \textrm{(Time independent)}
\end{cases}
\end{aligned}
\end{equation}

\subsection{X and acceleration; considerations of instant velocity}
\label{sec:org3fbee95}
According to  \ref{eq:NS-n1} and X definition, under \ref{eq:considerations}, we will have:
\begin{equation}
\begin{aligned}
\label{eq:NS-n1}
&(\rho{}\left[\frac{\partial{v}}{\partial{t}} + v\frac{\partial{v}}{\partial{x}} \right] = \dfrac{\partial \left(\mu \frac{\partial{v}}{\partial{x}} \right)}{\partial{x}} - \dfrac{\partial{p}}{\partial{x}} + X) \,\land\, (X = \rho{}. \dfrac{ (V(\rho) - v)}{\tau})\\
&\implies \frac{\textrm{d}v}{\textrm{d}t}=\frac{V(\rho) - v}{\tau}

\end{aligned}
\end{equation}
\end{document}