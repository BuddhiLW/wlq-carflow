% Created 2022-01-24 Mon 09:27
% Intended LaTeX compiler: pdflatex
\documentclass[12pt]{article}
\usepackage[utf8]{inputenc}
\usepackage[T1]{fontenc}
\usepackage{graphicx}
\usepackage{longtable}
\usepackage{wrapfig}
\usepackage{rotating}
\usepackage[normalem]{ulem}
\usepackage{amsmath}
\usepackage{amssymb}
\usepackage{capt-of}
\usepackage{hyperref}
\author{Pedro Branquinho}
\date{\today}
\title{}
\hypersetup{
 pdfauthor={Pedro Branquinho},
 pdftitle={},
 pdfkeywords={},
 pdfsubject={},
 pdfcreator={Emacs 27.2 (Org mode 9.6)}, 
 pdflang={English}}
\begin{document}

\tableofcontents


\section{{\bfseries\sffamily DONE} Done}
\label{sec:orgce7d909}
Pertubation Methods:
\begin{itemize}
\item Multiple Scales Expansion.
\item Anharmonic Oscillator example.
\end{itemize}
\url{https://www.math.arizona.edu/\~ntna2007/Perturbation\_Methods.pdf}

\section{NEXT Reading}
\label{sec:orge42d7ba}
Title: \textbf{PHY-892 The Many-Body problem, from
perturbation theory to dynamical-mean Öeld
theory (lecture notes)}.

Author: André-Marie Tremblay

\url{https://pitp.phas.ubc.ca/confs/sherbrooke2018/archives/N-corps-2017.pdf}

\section{{\bfseries\sffamily TODO} Todo}
\label{sec:orgfb48cac}
\url{http://galileoandeinstein.phys.virginia.edu/7010/CM\_22\_Resonant\_Nonlinear\_Oscillations.html}
\url{https://chem.libretexts.org/Bookshelves/Physical\_and\_Theoretical\_Chemistry\_Textbook\_Maps/Supplemental\_Modules\_(Physical\_and\_Theoretical\_Chemistry)}

\section{Spring-mass}
\label{sec:orgcc7162a}

Following \textbf{\href{https://www.uio.no/studier/emner/matnat/fys/FYS3120/v14/undervisningsmateriale/smalloscillations.pdf}{FYS 3120: Classical Mechanics and Electrodynamics}}, considering:
\subsection{The equation of motion}
\label{sec:org8302932}
\begin{equation}
\begin{aligned}
\ddot{q} = f(q, \dot{q})
\end{aligned}
\end{equation}

\subsubsection{Mass-spring}
\label{sec:org6e3345a}
In our case, mass-pring:

\begin{equation}
\begin{aligned}
m.\ddot{q} &= -k.q\\
\Leftrightarrow \ddot{q} &= - \frac{k}{m}q \, \land \, f(q,\dot{q})= - \frac{k}{m}q
\end{aligned}
\end{equation}

\subsection{Deviation from Equilibrium}
\label{sec:orga13ec44}
Let the variation from equilibrium be: \(\rho = q - q_0\).

\begin{equation}
\begin{aligned}
\ddot{\rho} = f(q_0 + \rho{}, \dot{\rho{}})
\end{aligned}
\end{equation}

e.g.,
\begin{equation}
\begin{aligned}
q = \rho + q_0 \implies \dot{q} = \dot{\rho} \,\land\,\ddot{q} = \ddot{\rho}
\end{aligned}
\end{equation}

\subsubsection{Mass-spring}
\label{sec:org30ebd78}
In our spring-mass case,

\begin{equation}
\begin{aligned}
\ddot{\rho} = - \frac{k}{m}(q_0 + \rho)
\end{aligned}
\end{equation}

\subsection{Power expansion - Expansion around \((q_0,0)\)}
\label{sec:orga6cd5da}

\begin{equation}
\begin{aligned}
\ddot{\rho} = f(q_0, 0) + \rho \dfrac{\partial{f}}{\partial{\rho}}(q_0,0) + \dot{\rho} \dfrac{\partial{f}}{\partial{\dot{\rho}}}(q_0,0)
\end{aligned}
\end{equation}

\(f(q_0,0)=0\, \because\, q_0\) is equilibrium point: \hyperref[sec:org8302932]{The equation of motion} would be zero in this point.

We also neglect second order or higher terms.

\subsubsection{Mass-spring}
\label{sec:orga71c2f8}
In our case,

\begin{equation}
\begin{aligned}
\ddot{\rho} = \rho \dfrac{\partial{f}}{\partial{\rho}}(q_0,0) + \dot{\rho} \dfrac{\partial{f}}{\partial{\dot{\rho}}}(q_0,0)\\
\end{aligned}
\end{equation}

in which,

\begin{equation}
\begin{aligned}
\left(\dfrac{\partial{f}}{\partial{\rho}}(\rho,\dot{\rho})
=- \dfrac{k}{m}\right) \, \land \, \left(\dfrac{\partial{\dot{f}}}{\partial{\dot{\rho}}}(\rho,\dot{\rho})=0\right)
\end{aligned}
\end{equation}

Therefore, in our case, the pertubation equation is:

\begin{equation}
\begin{aligned}
\ddot{\rho} = - \dfrac{k}{m}\rho
\end{aligned}
\end{equation}

\begin{equation}
\begin{aligned}
 \implies \rho(t) = A \sin{\left(\sqrt{\frac{k}{m}}t\right)} + B cos{\left(\sqrt{\frac{k}{m}}t\right)}
\end{aligned}
\end{equation}
\begin{enumerate}
\item Particular solution
\label{sec:org4181458}
Particular case if \(t_0 = 0\),

\begin{equation}
\begin{aligned}
\rho_0 &= B
\implies \rho(t) = \rho_0 \cos{\left(\sqrt{\frac{k}{m}}t\right)}
\end{aligned}
\end{equation}

\item General initical condition
\label{sec:org9f21614}

If \(\rho(t_0)=\rho_0\),

\begin{equation}
\begin{aligned}
\rho_0 &= A \sin{\left(\sqrt{\frac{k}{m}}t\right)} + B cos{\left(\sqrt{\frac{k}{m}}t\right)} \\
\Leftrightarrow  \rho_0 &= \left(\dfrac{A \sin{\left(\sqrt{\frac{k}{m}}t_0\right)} + B cos{\left(\sqrt{\frac{k}{m}}t_0\right)}}{\sqrt{A^2 + B^2}}\right). \sqrt{A^2 + B^2}
\end{aligned}
\end{equation}

Let the right-triangle with sides oposite side A and adjacent side B, with thus hippotenuse, \(\sqrt{A^2 + B^2}\). This triangle define a angle \(\alpha = \arctan{(\dfrac{A}{B})}\). So, \(\sin{(\alpha)}= \dfrac{A}{\sqrt{A^2 + B^2}}\) and \(\cos{(\alpha)}= \dfrac{B}{\sqrt{A^2 + B^2}}\).

\begin{equation}
\begin{aligned}
\implies \rho_0 &= (\sqrt{A^2 + B^2})\left(\sin{(\alpha)} \sin{\left(\sqrt{\frac{k}{m}}t_0\right)} + \cos{(\alpha)} cos{\left(\sqrt{\frac{k}{m}}t_0\right)}\right)\\
\therefore \rho_0 &= (\sqrt{A^2 + B^2}) \cos{\left(\sqrt{\frac{k}{m}}t_0 - \alpha\right)}
\end{aligned}
\end{equation}
\end{enumerate}
\end{document}
